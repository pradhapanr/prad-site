\documentclass[11pt, letterpaper]{article}
\usepackage{fullpage}
\usepackage{amsmath,amsthm,amsfonts,amssymb,amscd}
\usepackage{lastpage}
\usepackage{enumerate}
\usepackage{fancyhdr}
\usepackage{mathrsfs}
\usepackage{enumitem} % \setlist
% for \imageans: float for [H] so the figure floats
\usepackage{graphicx}
\usepackage{adjustbox}
\usepackage{float} 

\setlength{\parindent}{0.25in}
\setlength{\parskip}{0.05in}
% indent paragraphs in list
\setlist{  
  listparindent=\parindent,
  parsep=0pt,
}

% Include graphics in answer
\newcommand{\imageans}[1]
{%
    \begin{figure}[H]
        \centering
        \includegraphics[width=0.4\linewidth]{#1}
    \end{figure}
}

% comments inside align environnment
\newcommand{\comment}[1]{%
  \text{\phantom{(#1)}} \tag{#1}
}
\newtheorem{theorem}{Theorem}
\newtheorem{lemma}{Lemma}
% Cases for Proof environment
\newlist{pcases}{enumerate}{1}
\setlist[pcases]{
  label=\underline{Case~\arabic*}:\protect\thiscase.~,
  ref=\arabic*,
  align=left,
  labelsep=0pt,
  leftmargin=0pt,
  labelwidth=0pt,
  parsep=0pt
}
\newcommand{\case}[1][]{%
  \if\relax\detokenize{#1}\relax
    \def\thiscase{}%
  \else
    \def\thiscase{~#1}%
  \fi
  \item
}

% Edit these as appropriate
\newcommand\course{Math 3120}
\newcommand\hwtitle{HW 1}                  
\newcommand\name{David Tran}
\newcommand\studentid{251169871}

\fancypagestyle{firststyle}
{
    \headheight 35pt
    \lhead{\name}
    \lhead{\name\\\studentid}
    \chead{\textbf{\LARGE \hwtitle}}
    \rhead{\course \\ \today}
    \lfoot{}
    \cfoot{}
    \rfoot{\small\thepage}
    \headsep 1.5em
}

\DeclareUnicodeCharacter{2212}{-}
\begin{document}

\thispagestyle{firststyle}

\setlist[enumerate]{leftmargin=*} % remove enuemrate indentation

\begin{enumerate}
  \item \begin{proof}
    Let $x, g \in G$ a group. Suppose $\vert x \vert = n$ with $n$ finite. Then $x^n = 1$, so
    
    $$
    (g^{-1}xg)^n = (g^{-1}xg)\dots(g^{-1}xg) = g^{-1}x^ng = g^{-1}g = 1.
    $$
    If $m < n$ with $m \in \mathbb N$, then similarily $(g^{-1}xg)^m = g^{-1}x^mg$. Since $\vert x \vert = n$, $x^m \neq 1$, so $x^m g \neq g$ and $g^{-1}x^m g \neq 1$, since $g$ is the unique inverse of $g^{-1}$. So $\vert g^{-1}xg \vert = n = \vert x \vert$ when $n$ is finite. If $\vert x \vert = \infty$, then $\vert g^{-1}xg \vert = \infty$, for otherwise

    $$
    (g^{-1}xg)^k = g^{-1}x^kg = 1 \implies x^k = gg^{-1} = 1
    $$
    for some $k \in \mathbb N$, contradicting that $\vert x \vert = \infty$.

    Consequently, $\vert ab \vert = \vert a^{-1}(ab)a \vert = \vert ba \vert$.
  \end{proof}

  \item There are 3 "actions" to represent: a horizontal reflection, a vertical reflection, and a half-rotation. Represent these as
  \begin{align*}
    h &= (1 \enspace 4)(2 \enspace 3) \\
    v &= (1 \enspace 2)(3 \enspace 4) \\
    r &= (1 \enspace 3)(2 \enspace 4)
  \end{align*}
  respectively. Then $K := \lbrace 1, h, v, r \rbrace$, where $1$ is the identity function and $K \leq S_4$. The order of each element except the identity (which has order 1) is 2, since $h, v, r$ are composed entirely of 2-cycles. Finally, note that $K \leq S_4$ since $K \subseteq S_4$ is finite and closed under products, since $r^2 = h^2 = v^2 = 1$, $hr = v = rh$ and $vr = h = rv$.

  \item We use the fact that the order of a permutation is the least-common multiple of its cycles. Consider the lengths of the cycles in every permutation, its least-common multiple, and the number of such elements in $S_6$:
  
  \begin{align*}
    6 \implies 6,120 \\ 
    5 + 1 \implies 5,144 \\
    4 + 2 \implies 4,90 \\
    4 + 1 + 1 \implies 4,90 \\
    3 + 3 \implies 3,40 \\
    3 + 2 + 1 \implies 6,120 \\
    3 + 1 + 1 + 1 \implies 3,40 \\
    2 + 2 + 2 \implies 2,15 \\
    2 + 2 + 1 + 1 \implies 2,45\\
    2 + 1 + 1 + 1 + 1 \implies 2,15 \\
    1 + 1 + 1 + 1 + 1 + 1 \implies 1,1
  \end{align*}

  To the count the elements of each possible order, it suffices to count the number of permutations with each cycle pattern. To compute this, take the total number of permutations and divide out the product of the lengths of each cycle (since each $t$-cycle can be represented $t$ equivalent ways, i.e. $(1 \enspace 2) = (2 \enspace 1)$), as well as the number of ways to choose each cycle. For example, there are $\frac{6!}{(2^3)(3!)} = 15$ ways to permute 6 elements into a 3 2-cycles, since each 2-cycle can be represented 2 different ways, and there are $3!$ ways to choose the 2-cycles. Summing up the permutations that have identical least-common multiple cycle-lengths, we have there are 240 elements of order 6, 144 elements of order 5, 180 elements of order 4, 80 elements of order 3, 75 elements of order 2, and 1 element of order 1.

  \item \begin{proof}
    There is an element in $r \in D_{24}$ with order 12. The possible cycle lengths in every element of $S_4$ are $1 + 1 + 1 + 1, 2 + 1 + 1, 2 + 2, 3 + 1, 4$, none of which have a least-common multiple of 12. So there is no element of order 12 in $S_4$, so there can not be an isomorphism between $D_{24}$ and $S_4$, since isomorphisms preserve order.
  \end{proof}

  \item \begin{proof}
    $(\rightarrow)$ Suppose $\varphi$ is a homomorphism. Let $a, b \in G$. Then $\varphi(ab) = (ab)^2 = \varphi(a)(b) = a^2b^2$, so $abab = aabb \implies ba = ab$, by the Cancellation Law of $a$ on the left and $b$ on the right.

    $(\leftarrow)$ Suppose $G$ is Abelian. Let $a, b \in G$. Then $\varphi(ab) = (ab)^2 = a^2b^2 = \varphi(a)\varphi(b)$.
  \end{proof}

  \item \begin{proof}
    Let $a \in G$. We have $1 \cdot a = a1 = a$, so the identity property holds. Also, for $x, y \in G$, $x \cdot (y \cdot a) = x \cdot (ay^{-1}) = (ay^{-1})x^{-1} = a(y^{-1}x^{-1}) = xy \cdot a$, by associativity. So compatibility also holds, thus $g \cdot a$ is a group action. 
  \end{proof}

  \item \begin{proof}
    $a \sim a$ since $a = 1a$, so reflexivity holds. If $a \sim b$, then $a = gb$ for $g \in G$, so $b = g^{-1}a$ and $b \sim a$, so symmetry holds. Finally, suppose $a \sim b$ and $b \sim c$. Then $a = gb$ and $b = hc$ for some $g, h \in G$. We have $a = gb = ghc = (gh)c$, so transitivity holds. So $\sim$ is an equivalence relation.
  \end{proof}

  \item \begin{proof}
    First we show that for any group $G$ acting on $S$, its stabilizer $G_s$ is a subgroup of $G$. We have $1 \in G_s$ by the axiom of group actions. Also, $G_s$ is closed under inverses since if $x \in G_s$, then $s = x^{-1}x \cdot s = x^{-1} \cdot (xs) = x^{-1}s$. It is also closed under products since if $x, y \in G_s$, then $(xy) \cdot s = x \cdot (y \cdot s) = x \cdot s = s$. So $G_s \leq G$.

    Finally we prove the proposition: note that $\sigma \cdot s = \sigma(s)$ is a group action of $G = S_n$ on $\lbrace 1, 2, \dots , n \rbrace$, since $\operatorname{id}(s) = s$ and $\sigma(\tau(s)) = (\sigma \circ \tau)(s)$, where $\sigma, \tau \in G$, by composition of functions. So $G_i$ is a stabilizer of $G$, thus $G_i \leq G$.
  \end{proof}

\end{enumerate}
\end{document}
