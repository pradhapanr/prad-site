\documentclass[11pt, letterpaper]{article}
\usepackage{fullpage}
\usepackage{amsmath,amsthm,amsfonts,amssymb,amscd}
\usepackage{lastpage}
\usepackage{enumerate}
\usepackage{fancyhdr}
\usepackage{mathrsfs}
\usepackage{enumitem} % \setlist
% for \imageans: float for [H] so the figure floats
\usepackage{graphicx}
\usepackage{adjustbox}
\usepackage{float} 

\setlength{\parindent}{0.25in}
\setlength{\parskip}{0.05in}
% indent paragraphs in list
\setlist{  
  listparindent=\parindent,
  parsep=0pt,
}

% Include graphics in answer
\newcommand{\imageans}[1]
{%
    \begin{figure}[H]
        \centering
        \includegraphics[width=0.4\linewidth]{#1}
    \end{figure}
}

% comments inside align environnment
\newcommand{\comment}[1]{%
  \text{\phantom{(#1)}} \tag{#1}
}
\newtheorem{theorem}{Theorem}
\newtheorem{lemma}{Lemma}
% Cases for Proof environment
\newlist{pcases}{enumerate}{1}
\setlist[pcases]{
  label=\underline{Case~\arabic*}:\protect\thiscase.~,
  ref=\arabic*,
  align=left,
  labelsep=0pt,
  leftmargin=0pt,
  labelwidth=0pt,
  parsep=0pt
}
\newcommand{\case}[1][]{%
  \if\relax\detokenize{#1}\relax
    \def\thiscase{}%
  \else
    \def\thiscase{~#1}%
  \fi
  \item
}

% Edit these as appropriate
\newcommand\course{Math 3120}
\newcommand\hwtitle{HW 4}                  
\newcommand\name{David Tran}
\newcommand\studentid{251169871}

\fancypagestyle{firststyle}
{
    \headheight 35pt
    \lhead{\name}
    \lhead{\name\\\studentid}
    \chead{\textbf{\LARGE \hwtitle}}
    \rhead{\course \\ \today}
    \lfoot{}
    \cfoot{}
    \rfoot{\small\thepage}
    \headsep 1.5em
}

\DeclareUnicodeCharacter{2212}{-}
\begin{document}

\thispagestyle{firststyle}

\setlist[enumerate]{leftmargin=*} % remove enuemrate indentation

\begin{enumerate}
  \item \begin{enumerate}
    \item The conjugacy classes of $Q_8$ are $\lbrace 1 \rbrace, \lbrace \pm i \rbrace, \lbrace \pm j \rbrace, \lbrace \pm k \rbrace$.
    \item The conjugacy classes of $A_4$ are identified by their cycle type: \begin{gather*}
      \lbrace 1 \rbrace \\
      \lbrace (12)(34), (13)(24), (14)(23) \rbrace \\
      \lbrace (123), (341), (243), (421), \rbrace \\
      \lbrace (132), (412), (234), (314) \rbrace
    \end{gather*}
  \end{enumerate}

  \item \begin{enumerate}
    \item \begin{proof}
      Recall that by Proposition 4.6 the size of the conjugacy class of $x$ is the index of its normalizer. $Z(G) \leq C_G(x)$ so $\vert G : C_G(x) \vert \leq \vert G : Z(G) \vert = n$.
    \end{proof}
    \item \begin{proof}
      Let $G$ be a group with exactly two conjugacy classes. Since $\lbrace 1 \rbrace$ forms its own conjugacy class, let $\mathcal C$ be the other distinct conjugacy class. By the class equation, $\vert G \vert = 1 + \vert \mathcal C \vert$. Since $\vert \mathcal C \vert \mid \vert G \vert$, we must have $\vert \mathcal C \vert = 1$. So $\vert G \vert = 2$. Thus, only the groups of cyclic order 2 have exactly two conjugacy classes.
    \end{proof}
  \end{enumerate}

  \item \begin{proof}
    We note that an element $x$ is in the center of a group $G$ if and only if the order of its conjugacy class is 1, since $gx = xg \implies gxg^{-1} = x$. It suffices to show then that every non-identity element of $S_n$ has a conjugacy class of order greater than 1. Since in $S_n$ for $n \geq 3$, there is more than one distinct $m$-cycle for $m \leq n$, for any $\sigma \in S_n$ with cycle decomposition $\sigma = \tau_1\dots\tau_n$, we can find a distinct $\sigma' \in S_n$ with the same cycle type (and thus in the same conjugacy class) as $\sigma$ by choosing a distinct cycle of the same length for each cycle $\tau_1, \dots, \tau_n$.
  \end{proof}
  \item \begin{proof}
    Since $H$ is normal, $N_G(H) = G$. So by Corollary 15, $G/C_G(H)$ is isomorphic to some subgroup of $\operatorname{Aut}(H)$. By Proposition 17, $\operatorname{Aut}(H) = \vert H \vert - 1 = 6$ since $\vert H \vert$ is prime, so $\vert G/C_G(H) \vert \mid 6$. But, 7 is the smallest prime dividing $\vert G \vert = 203$, so $\vert G/C_G(H) \vert = 1$, thus $G = C_G(H)$. So $H \leq Z(G)$. If $H < Z(G)$, then $Z(G) = G$ so $G$ is abelian. If instead $H = Z(G)$, then $G/H$ is cyclic, so $G$ is abelian.
  \end{proof}
  \item \begin{enumerate}
    \item \begin{proof}
      Suppose $H$ char $K$ and $K \trianglelefteq G$. Since $K \trianglelefteq G$, every inner automorphism of $G$ restricted to $K$ is an automorphism of $K$ by Proposition 4.13. Since $H$ char $K$, every automorphism of $K$ maps $H$ to itself. In particular, the inner automorphism of $G$ maps $H$ to itself, that is, $gHg^{-1} = H$ for all $g \in G$. So $H$ is normal.
    \end{proof}
    \item \begin{proof}
      Let $\varphi \in \operatorname{Aut}(G)$. Since $K$ char $G$, $\varphi(K) = K$. So $\varphi \in \operatorname{Aut}(K)$, and since $H$ char $K$, $\varphi(H) = H$. Thus $H$ char $K$.
    \end{proof}
  \end{enumerate}

  \item \begin{enumerate}
    \item Since $12 = 2^2 \cdot 3$, the Sylow 2-subgroups are the subgroups of order 4. The elements of order 2 in $D_{12}$ are $\lbrace 1, r^3, s, sr, sr^2, sr^3, sr^4, sr^5 \rbrace$. So the Sylow 2-subgroups are $\langle s, r^3 \rangle$, $\langle r^3, sr \rangle$, $\langle r^3, sr^2 \rangle$ (these comprise all non-identity elements of order 2 and are conjugates).
    
    Similarily, the Sylow 3-subgroups are the subgroups of order 3. The only elements of orders 1 or 3 are $\lbrace 1, r^2, r^4 \rbrace$, so this is the sole Sylow 3-subgroup of $D_{12}$.

    \item Since $\vert S_4 \vert = 2^3 \cdot 3$, the Sylow 2-subgroups then are the subgroups of order 8. Let $G$ be a subgroup of $S_4$ isomorphic to $D_8$ (by Cayley's Theorem). Then the conjugations of $G$ are the Sylow 2-subgroups: namely the symmetries of a square with vertices labelled $\lbrace 1,2,3,4 \rbrace$, $\lbrace 1,2,4,3 \rbrace$, and $\lbrace 1,3,2,4 \rbrace$. That is, $\langle (1234), (12)(34) \rangle, \langle (1243), (12)(43) \rangle$, and $\langle (1324) (13)(24) \rangle$. By Sylow's Theorem, $n_2 \equiv 1 \pmod 2$ and $n_2 \mid 3$, so these are all of them.
    
    The Sylow 3-subgroups are the subgroups of order 3, which are generated by the 3-cycles in $S_4$: $\langle (123) \rangle, \langle (134) \rangle, \langle (234) \rangle, \langle (124) \rangle$.
  \end{enumerate}

  \item \begin{enumerate}
    \item $105 = 3 \cdot 7 \cdot 5$. By Sylow's Theorem, $n_{7} \equiv 1 \pmod 7$ so $n_7 = 1, 8, 16, \dots$ and $n_{7} \mid 15$, so $n_7 = 1$. Thus there is one Sylow 7-subgroup, so it is normal, by Corollary 20.
    \item $351 = 3^3 \cdot 13$. By Sylow's Theorem, $n_3 = 1 \pmod 3$ and $n_3 \mid 13$. So $n_3 = 13$ or 1. If $n_3 = 1$ we are done so suppose $n_3 = 13$. Since each Sylow 3-subgroup has $3^3 = 27$ elements, there are $13 \cdot 26 = 338$ distinct non-identity elements with orders that divide 27. This leaves $351 - 338 = 13$ distinct elements. A Sylow 13-subgroup must have order $13^1 = 13$ since the prime factor decomposition of 351 has only 1 13 term, so the 13 distinct elements form the unique Sylow 13-subgroup. So it is normal.
  \end{enumerate}

  \item \begin{proof}
    Let $G$ be a simple group of order $168 = 7 * 3 * 2^3$. There are $n_7 \equiv 1 \pmod 7$ Sylow 7-subgroups with $n_7 \mid 168$. Since $168 / 7 = 24$, $n_7 \leq 24$. Since $G$ is simple, the Sylow 7-subgroups are not normal, so $n_7 > 1$. Thus $n_7 = 8$, so there are $6 \cdot 8 = 48$ elements of order 7 (note that the order of each element except the identity in each Sylow 7-subgroup is 7, since 7 is prime).
  \end{proof}
  
  \item $n_5 = 6$ since $n_5 \pmod 5$ and $n_5 \mid 60$ and $1 < n_5 \leq 60/5 = 12$, since $A_5$ is simple. Similarily, $n_3$ is 10 or 4. But since there are $(5)(4)(3)/3 = 30$ 3-cycles, and each 3-cycle is in a Sylow 3-subgroup, we must have $n_3 = 10$. We've used $(3 - 1)(10) + (4)(6) = 44$ non-identity elements, leaving $15$ non-identity elements left. Finally, $n_2 = 3, 5, 15$ by the same reasoning as above. Since $A_5$ consists only of 3-cycles, 5-cycles, and the product of 2 2-cycles, the 15 non-identity elements left are products of 2 2-cycles. We can generate $15/3 = 5$ groups of order $2^2 = 4$ with these non-identity elements, so $n_2 = 5$.
  
  \item First a lemma (Exercise 42 in Chapter 3):
  \begin{lemma}Let $H, K \trianglelefteq G$ with $H \cap K = 1$. Then $xy = yx$ for all $x \in H$ and $y \in K$.
  \end{lemma}
  \begin{proof}
    Let $x \in H, y \in K$. Then $x^{-1}y^{-1}xy = x^{-1}h \in H$ where $h = y^{-1}xy \in H$ by normality of $H$. Similarily, $x^{-1}y^{-1}xy = ky \in K$ where $k = x^{-1}y^{-1}x \in K$ by normality of $K$. So $x^{-1}y^{-1}xy \in H \cap K = 1$, so $xy = yx$. 
  \end{proof}

  Now we prove the theorem.

  \begin{proof}
    Let $H$ be a proper, non-trivial normal subgroup of $S_n$. Since $A_n \trianglelefteq S_n$, we have $H \cap A_n \trianglelefteq S_n$. Since $H \cap A_n \leq A_n$, $H \cap A_n \trianglelefteq A_n$. But, $A_n$ is simple, so either $H \cap A_n = 1$ or $H \cap A_n = A_n$. If $H \cap A_n = A_n$ then, $H \leq A_n \leq S_n$, and because $[S_n : A_n] = [S_n : H][H : A_n] = 2$, one of the indices is 1, so either $H = A_n$ or $H = S_n$, and we are done. So suppose instead $H \cap A_n = 1$. By Lemma 1, $H \subseteq Z(S_n) = 1$ by Question 3. So $H = 1$.
  \end{proof}
\end{enumerate}
\end{document}
