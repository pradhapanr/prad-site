\documentclass[11pt, letterpaper]{article}
\usepackage{fullpage}
\usepackage{amsmath,amsthm,amsfonts,amssymb,amscd}
\usepackage{lastpage}
\usepackage{enumerate}
\usepackage{fancyhdr}
\usepackage{mathrsfs}
\usepackage{enumitem} % \setlist
% for \imageans: float for [H] so the figure floats
\usepackage{graphicx}
\usepackage{adjustbox}
\usepackage{float} 

\setlength{\parindent}{0.25in}
\setlength{\parskip}{0.05in}
% indent paragraphs in list
\setlist{  
  listparindent=\parindent,
  parsep=0pt,
}

% Include graphics in answer
\newcommand{\imageans}[1]
{%
    \begin{figure}[H]
        \centering
        \includegraphics[width=0.4\linewidth]{#1}
    \end{figure}
}

% comments inside align environnment
\newcommand{\comment}[1]{%
  \text{\phantom{(#1)}} \tag{#1}
}
\newtheorem{theorem}{Theorem}
\newtheorem{lemma}{Lemma}
% Cases for Proof environment
\newlist{pcases}{enumerate}{1}
\setlist[pcases]{
  label=\underline{Case~\arabic*}:\protect\thiscase.~,
  ref=\arabic*,
  align=left,
  labelsep=0pt,
  leftmargin=0pt,
  labelwidth=0pt,
  parsep=0pt
}
\newcommand{\case}[1][]{%
  \if\relax\detokenize{#1}\relax
    \def\thiscase{}%
  \else
    \def\thiscase{~#1}%
  \fi
  \item
}

% Edit these as appropriate
\newcommand\course{Math 3122}
\newcommand\hwtitle{HW 2}                  
\newcommand\name{David Tran}
\newcommand\studentid{251169871}

\fancypagestyle{firststyle}
{
    \headheight 35pt
    \lhead{\name}
    \lhead{\name\\\studentid}
    \chead{\textbf{\LARGE \hwtitle}}
    \rhead{\course \\ \today}
    \lfoot{}
    \cfoot{}
    \rfoot{\small\thepage}
    \headsep 1.5em
}

\DeclareUnicodeCharacter{2212}{-}
\begin{document}

\thispagestyle{firststyle}

\setlist[enumerate]{leftmargin=*} % remove enuemrate indentation

\begin{enumerate}
  \item \begin{enumerate}
    \item \begin{proof} Let $\epsilon > 0$. Choose $N \in \mathbb N$ such that $\frac{1}{2N} < \epsilon$. Note that $\int_0^1 f_n(x) dx = \frac{n - 1}{2n}$ and $f_m(x) \geq f_n(x)$ if $m > n$. So, for $m, n \geq N$, where $m \geq n$,
    \begin{align*}
      \int_0^1 \vert f_m(x) - f_n(x) \vert dx
      &= \int_0^1 f_m(x) dx - \int_0^1 f_n(x) dx \\
      &= \frac{m - n}{2mn} \\
      &= \frac{m - n}{2(n + (m - n))n} \\
      &\leq \frac{n - m}{2(m - n)n} \\
      &= \frac{1}{2n} \\
      &\leq \frac{1}{2N} \\
      &< \epsilon
    \end{align*}
    So $(f_n)$ is Cauchy in $(C[0, 1], d)$.
    \end{proof}

    To show that $(f_n)$ diverges in $(C[0, 1], d)$, we first show that $(f_n)$ converges to the step function 
    $$
      g(x) = \begin{cases}0, & 0 \leq x < 1/2 \\ 1, & 1/2 \leq x \leq 1\end{cases}
    $$ 
    in $(\mathbb R^{[0, 1]}, d)$, where $\mathbb R^{[0, 1]}$ is the set of functions from $[0, 1] \to \mathbb R$.
    \begin{proof}
      Let $\epsilon > 0$. Choose $N \in \mathbb N$ such that $\frac{1}{2N} < \epsilon$. Since $g(x) \geq f_n(x)$ for all $n \in \mathbb N$, 
      $
        d(f_n, g) = \int_0^1 g(x) dx - \int_0^1 f_n(x) dx.
        = \frac{1}{2} - \frac{n - 1}{2n}
        = \frac{1}{2n}
        \leq \frac{1}{2N}
        < \epsilon
      $.
    \end{proof}
    Since $g \not\in C[0, 1]$ (the step function is discontinuous at $x = 1/2$), by uniqueness of convergence, $(f_n)$ does not converge in $(C[0, 1], d)$.
    \item \begin{proof}
      Choose $\epsilon = 1/2$. Let $N \in \mathbb N$. Choose $m, n \in \mathbb N$ such that $m = 2n > n > N$. Then, for $x \in \left(\frac{1}{2}, \frac{1}{2} + \frac{1}{m} \right]$,
      \begin{align*}
        \vert f_m(x) - f_n(x) \vert 
        &= mx - \frac{m}{2} - \left(nx - \frac{n}{2}\right) \\
        &= 2nx - n - nx + \frac{n}{2} \\
        &= nx - \frac{n}{2}
      \end{align*}
      so $\left\vert f_m\left(\frac{1}{2} + \frac{1}{m}\right) - f_n\left(\frac{1}{2} + \frac{1}{m}\right) \right\vert = \left\vert f_m\left(\frac{1}{2} + \frac{1}{2n}\right) - f_n\left(\frac{1}{2} + \frac{1}{2n}\right) \right\vert = \frac{1}{2}$. Thus, $\sup_{x \in [0, 1]} \vert f_m(x) - f_n(x) \vert \geq \frac{1}{2} = \epsilon$. So $(f_n)$ is not Cauchy.
    \end{proof}
    \newpage
  \end{enumerate}
  \item \begin{proof} First we show $d_1$ is a metric.
  \begin{enumerate}
    \item $d_1((x, y), (x', y')) \geq 0$ since $d_X(x, x') \geq 0$ and $d_Y(y, y') \geq 0$.
    \item $d_1((x, y), (x, y)) = d_X(x, x) + d_Y(y, y) = 0$.
    \item $d_1((x, y), (x', y')) = d_X(x, x') + d_Y(y, y') = d_X(x', x) + d_Y(y', y) = d_1((x', y'), (x, y))$.
    \item $d_1((x, y), (x', y')) + d_1((x', y'), (x'', y'')) = d_X(x, x') + d_Y(y, y') + d_X(x', x'') + d_Y(y', y'') \geq d_X(x, x'') + d_Y(y, y'') = d_1((x, y), (x'', y''))$
  \end{enumerate}

  Next we show $d_\infty$ is a metric.
  \begin{enumerate}
    \item $d_\infty((x, y), (x', y')) \geq 0$ since $d_X(x, x'), d_Y(y, y') \geq 0$.
    \item $d_\infty((x, y), (x, y)) = \max \lbrace d_X(x, x), d_Y(y, y) \rbrace = 0$.
    \item $d_\infty((x, y), (x', y')) = \max \lbrace d_X(x, x'), d_Y(y, y') \rbrace = \max \lbrace d_X(x', x), d_Y(y', y) \rbrace = d_\infty((x', y'), (x, y))$.
    \item Note that if $d_X(x, x') \geq d_Y(y, y')$ and $d_X(x', x'') \geq d_Y(y', y'')$, then $L = d_\infty((x, y), (x', y')) + d_\infty((x', y'), (x'', y'')) = d_X(x, x') + d_X(x', x'') \geq d_X(x, x'')$. Similarily, if $d_Y(y, y') \geq d_X(x, x')$ and $d_Y(y', y'') \geq d_X(x, x'')$, then $L \geq d_Y(y, y'')$. If $d_X(x, x') \geq d_Y(y, y')$ and $d_Y(y', y'') \geq d_X(x', x'')$, then $L = d_X(x, x') + d_Y(y', y'') \geq d_X(x, x') + d_X(x', x'') \geq d_X(x, x'')$. Finally if $d_Y(y, y') \geq d_X(x, x')$ and $d_X(x', x'') \geq d_Y(y', y'')$, then $L = d_Y(y, y') + d_X(x', x'') \geq d_X(x, x'')$. So, we have $L \geq d_X(x, x''), d_Y(y, y'')$; that is, $L \geq \max \lbrace d_X(x, x''), d_Y(y, y'') \rbrace = d_\infty((x, y), (x'', y''))$.
  \end{enumerate}
  \end{proof}

  \begin{proof}
    $(\rightarrow)$ Let $S \subseteq (X \times Y)$ be open with respect to $d_1$. Let $p = (p_x, p_y) \in S$. Then, there is an $\epsilon > 0$ such that $B_{d_1}(p; \epsilon) \subseteq S$. Let $(x, y) \in B_{d_\infty}(p; \epsilon/2)$. Then $d_X(x, p_x), d_Y(y, p_y) \leq \epsilon/2$, so $d_X(x, p_x) + d_Y(y, p_y) \leq \epsilon$ and $(x, y) \in B_{d_1}(p; \epsilon)$. So $B_{d_\infty}(p; \epsilon/2) \subseteq B_{d_1}(p; \epsilon) \subseteq S$, so $S$ is open with respect to $d_\infty$. 

    $(\leftarrow)$ Let $S$ be open with respect to $d_\infty$. Let $p = (p_x, p_y) \in S$. Then, there is an $\epsilon > 0$ such that $B_{d_\infty}(p; \epsilon) \subseteq S$. Let $(x, y) \in B_{d_1}(p; \epsilon)$. Then, $d_X(x, p_x) + d_Y(y, p_y) \leq \epsilon$, so $d_X(x, p_x), d_Y(y, p_y) \leq \epsilon$. So $\max \lbrace d_X(x, p_x), d_Y(y, p_y) \rbrace \leq \epsilon$, and $(x, y) \in B_{d_\infty}(p; \epsilon)$. So $B_{d_1}{p; \epsilon} \subseteq B_{d_\infty}(p; \epsilon) \subseteq S$, so $S$ is open with respect to $d_1$.
  \end{proof}
\end{enumerate}

\end{document}
