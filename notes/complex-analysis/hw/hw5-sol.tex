\documentclass[11pt, letterpaper]{article}
\usepackage{fullpage}
\usepackage{amsmath,amsthm,amsfonts,amssymb,amscd}
\usepackage{lastpage}
\usepackage{enumerate}
\usepackage{fancyhdr}
\usepackage{mathrsfs}
\usepackage{enumitem} % \setlist
% for \imageans: float for [H] so the figure floats
\usepackage{graphicx}
\usepackage{adjustbox}
\usepackage{float} 

\setlength{\parindent}{0.25in}
\setlength{\parskip}{0.05in}
% indent paragraphs in list
\setlist{  
  listparindent=\parindent,
  parsep=0pt,
}

% Include graphics in answer
\newcommand{\imageans}[1]
{%
    \begin{figure}[H]
        \centering
        \includegraphics[width=0.4\linewidth]{#1}
    \end{figure}
}

% comments inside align environnment
\newcommand{\comment}[1]{%
  \text{\phantom{(#1)}} \tag{#1}
}
\newtheorem{theorem}{Theorem}
\newtheorem{lemma}{Lemma}
% Cases for Proof environment
\newlist{pcases}{enumerate}{1}
\setlist[pcases]{
  label=\underline{Case~\arabic*}:\protect\thiscase.~,
  ref=\arabic*,
  align=left,
  labelsep=0pt,
  leftmargin=0pt,
  labelwidth=0pt,
  parsep=0pt
}
\newcommand{\case}[1][]{%
  \if\relax\detokenize{#1}\relax
    \def\thiscase{}%
  \else
    \def\thiscase{~#1}%
  \fi
  \item
}

% Edit these as appropriate
\newcommand\course{Math 3124}
\newcommand\hwtitle{HW 5}                  
\newcommand\name{David Tran}
\newcommand\studentid{251169871}

\fancypagestyle{firststyle}
{
    \headheight 35pt
    \lhead{\name}
    \lhead{\name\\\studentid}
    \chead{\textbf{\LARGE \hwtitle}}
    \rhead{\course \\ \today}
    \lfoot{}
    \cfoot{}
    \rfoot{\small\thepage}
    \headsep 1.5em
}

\DeclareUnicodeCharacter{2212}{-}
\begin{document}

\thispagestyle{firststyle}

\setlist[enumerate]{leftmargin=*} % remove enuemrate indentation

\begin{enumerate}
  \item First we prove the following lemma.
  \begin{lemma}
    Let $\lbrace z_n \rbrace \subseteq \mathbb C$ be infinite and bounded. Then there exists a subsequence $\lbrace z_{n_k} \rbrace \subseteq \lbrace z_n \rbrace$ convergent to some point $z \in \mathbb C$.
  \end{lemma}
  \begin{proof}
    Let $\lbrace z_n \rbrace$ be bounded. It suffices to show that a subsequence of $\lbrace z_n \rbrace$ where both the real and imaginary parts converge. Since $\lbrace z_n \rbrace$ is bounded, $\lbrace \Re(z_n) \rbrace$ is bounded, so by the lemma, there exists a subsequence $\lbrace z_{n_k} \rbrace$ such that the real parts converge. Since $\lbrace z_{n_k} \rbrace \subseteq \lbrace z_n \rbrace$, it is also bounded, so again by Bolzano-Weierstrass $\lbrace \Im(z_{n_k}) \rbrace$ is bounded and there exists a subsequence $\lbrace z_{n_{k_l}} \rbrace \subseteq \lbrace z_{n_k} \rbrace$ such that the imaginary parts converge. Thus the real and imaginary parts of $\lbrace z_{n_{k_l}} \rbrace$ converge, so the subsequence is convergent.
  \end{proof}

  Now we prove the theorem.
  
  \begin{proof}
    Let $f$ be a continuous complex-valued function on a compact set $K \subseteq \mathbb C$. Set $M := \sup_{z \in K} \vert f(z) \vert$. Consider the sequence $\lbrace f(z_n) \rbrace$ defined such that $M - \frac{1}{n} < \vert f(z_n) \vert$ for all $n \geq 1$, which exists by definition of supremum. Since $\vert f(z_n) \vert \leq M$ for all $z_n \in K$, the sequence converges to an $\alpha$ with $\vert \alpha \vert = M$. Since $K$ is compact, it is bounded, so by Bolzano-Weierstrass, there exists a subsequence $\lbrace z_{n_k} \rbrace \subseteq \lbrace z_n \rbrace$ convergent to some point $z \in \mathbb C$. Since $f$ is continuous, $\lbrace f(z_{n_k}) \rbrace$ also converges, and since $\lbrace f(z_{n_k}) \rbrace \subseteq \lbrace f(z_n) \rbrace$, it converges to $\alpha$. Finally, since $K$ is compact and $z$ is a limit point of $K$, $z \in K$. Thus $\alpha = f(z)$.

    Note that in the above, we assumed $M$ was finite. This can be shown as follows. Suppose otherwise. Then $f(K)$ is unbounded. But this contradicts that $f$ is continuous, since continuous functions map compact sets to compact sets. 
  \end{proof}
  % \item \begin{proof}
  %   Let $f$ be a continuous complex-valued function on a compact set $K \subseteq \mathbb C$. Define $M := \sup_{z \in K} \vert f(z) \vert$. We show that $M$ is finite. Suppose otherwise. 
  % \end{proox2f}
  % \begin{enumerate}
  %   \item \begin{proof}
  %     Suppose $M$ is finite. Then suppose there exists a $z \in K$ such that $\vert f(z) \vert = M$. Then the constant sequence $\lbrace f(z) \rbrace$ converges to an $\alpha \in f(K)$ such that $\vert \alpha \vert = M$ trivially. So suppose otherwise. Then by the definition of supremum, for all $n \in \mathbb N$, there is a $z_n \in K$ such that $M - \frac{1}{n} < \vert f(z_n) \vert$. We show that $\lim_{n \to \infty} f(z_n) = \alpha$ for some $\alpha$ with $\vert \alpha \vert = M$. Let $\epsilon > 0$. Choose $N \in \mathbb N$ such that for all $n \geq N$, $\frac{1}{n} < \epsilon$. Then,

  %     \begin{align*}
  %       \vert \alpha - f(z_n) \vert 
  %       &= \vert \alpha + (-f(z_n)) \vert \\
  %       &\leq \vert \alpha \vert + \vert -f(z_n) \vert \\
  %       &= \vert \alpha \vert + \vert f(z_n) \vert \\
  %       &\leq \vert \alpha \vert - \vert f(z_n) \vert \\
  %       &= M - \vert f(z_n) \vert \\
  %       &< \frac{1}{n} \\
  %       &< \epsilon
  %     \end{align*}

  %     Now suppose $M$ is infinite. Then for all $r \in \mathbb \mathbb R$, there is a $z_n \in K$ such that $\vert f(z_n) \vert > r$. Then $\lim_{n \to \infty} \vert f(z_n) \vert = \infty$.
  % \end{proof}

  \item \begin{proof}
    Let $f$ be entire. Recall that in the power series expansion $f(z) = \sum C_k z^k$, $C_k = \frac{f^{(k)}(0)}{k!}$. From the proof of Theorem 5.5, we found that $C_k = \frac{1}{2\pi i} \int_C \frac{f(w)}{w^{k + 1}} dw$. where $C$ is a circle centered at the origin of radius $\vert w \vert$ containing $z$. Setting the two equal, we have $f^{(k)}(0) = \frac{k!}{2\pi i} \int_C \frac{f(w)}{w^{k + 1}} dw$ for $k = 1, 2, \dots$. Define $g(z) := f(z + a)$. Then 
    \begin{align*}
      f^{(k)}(a) &= g^{(k)}(0) = \frac{k!}{2\pi i} \int_C \frac{g(\omega)}{\omega^{k + 1}} d\omega \\
      &= \frac{k!}{2\pi i} \int_C \frac{f(\omega + a)}{\omega^{k + 1}} d\omega \\
      &= \frac{k!}{2\pi i} \int_C \frac{f(w)}{(w - a)^{k + 1}} dw \\
    \end{align*}
    using the parameterization $\omega = w - a$.
  \end{proof}

  \item \begin{enumerate}
    \item \begin{proof}
      Suppose $f$ is entire with $\vert f \vert \leq M$ along $\vert z \vert = R$. From above, we have that the coefficients of the power series expansion of $f$ about 0 is given by $C_k = \frac{1}{2 \pi i} \int_C \frac{f(w)}{w^{k + 1}} dw$, where $C$ is a circle centered at the origin of radius $\vert w \vert$ containing $z$. Then, the integrand is bounded above by $\frac{M}{R^{k + 1}}$. The length of $C$ is $2\pi R$, so by the M-L Theorem, $\vert C_k \vert \leq \vert \frac{1}{2\pi i} \left(\frac{M}{R^{k + 1}}\right)(2 \pi R) \vert = \frac{M}{R^k}$.
    \end{proof}

    \item \begin{proof}
      Polynomial functions are entire, so by above, with $M = 1$ and $R = 1$, $\vert C_k \vert \leq \frac{M}{R^k} = 1$.
    \end{proof}
  \end{enumerate}

  \item \begin{proof}
    Let $f$ be entire with $\vert f(z) \vert \leq A + B \vert z \vert^k$. Then, along $\vert z \vert = R$, $\vert f \vert \leq A + BR^k$. From above, $\vert C_j \vert \leq \frac{A + BR^k}{R^j}$. Now suppose $j > k$. Taking circles of radius $R$ of arbitrary size, $\lim_{R \to \infty} \frac{A + BR^k}{R^j} = 0$, since $j > k$.
  \end{proof}

  \item \begin{proof}
    Let $f$ be entire with $\vert f(z) \vert \leq A + B \vert z \vert^{3/2}$. From above, $\vert C_k \vert = 0$ for $k \geq 3/2$. That is, $\vert C_k \vert \neq 0$ only for $k = 0, 1$. So $f(z) = C_0 + C_1 z$.
  \end{proof}

  \item \begin{proof}
    Suppose $f$ is entire. Since it is entire, it is continuous, and thus bounded on the compact set $0 \leq x, y \leq 1$. By periodicity, it is also bounded on all 1-by-1 squares $a \leq x, y \leq a + 1$ where $a \in \mathbb Z$. So, the function is bounded on the entire complex plane, so it is constant, by Liouville's Theorem.
  \end{proof}

  \item \begin{proof}
    $(\rightarrow)$ Suppose $P(z) = (z - \alpha)^kQ(z)$ where $Q$ is analytic and $Q(\alpha) \neq 0$. Then
    $$
    P'(z) = k(z - \alpha)^{k - 1}Q(z) + (z - \alpha)^kQ'(z) = (z - \alpha)^{k - 1}(kQ(z) + (z - \alpha)Q'(z))
    $$
    So $P'(z) = (z - \alpha)^{k - 1}Q_1(z)$ with $Q_1(\alpha) \neq 0$. Repeating as in the first part of the proof, $P^{(n)}(z) = (z - \alpha)^{k - n}Q_n(z)$ with $Q_n(\alpha) \neq 0$. Note that $P^{(k)}(z) = k!Q_k(z) \neq 0$ since $Q_k(z) \neq 0$. So $P^{(0)}(\alpha) = P^{(1)}(\alpha) = \dots = P^{(k - 1)}(\alpha) = 0$, and $P^{(k)}(\alpha) \neq 0$.

    $(\leftarrow)$ Suppose $P(\alpha) = P'(\alpha) = \dots = P^{(k - 1)}(\alpha) = 0$ and $P^{(k)}(\alpha) \neq 0$. We show that $P(z) = (z - \alpha)^kQ(z)$ where $Q$ is analytic and $Q(\alpha) \neq 0$. First note that since $P^{(k - 1)}(\alpha) = 0$, $P^{(k - 1)}(z) = (z - \alpha)Q(z)$ by the Factor Theorem. Also, $Q(\alpha) \neq 0$ since \begin{align*}
      P^{(k)}(z) = Q(z) + (z - \alpha)Q'(z)
    \end{align*}
    and $P^{(k)}(z) \neq 0$ by assumption. We finish with induction. Suppose $P^{(j)}(z) = (z - \alpha)^nQ(z)$ with $Q$ analytic and $Q(\alpha) \neq 0$, and $j < k$ and $n \geq 1$. We show that $P^{(j - 1)}(z)$ has root $\alpha$ with multiplicity $n + 1$. From the initial assumption, $P^{(j - 1)}(z) = (z - \alpha)S(z)$ with $S$ analytic. Differentiating gives 
    $$
    P^{(j)}(z) = (z - \alpha)^{n + 1}Q(z) = S(z) + (z - \alpha)S'(z),
    $$
    by the inductive hypothesis. so $S(z)$ has multiplicity $n$. Thus $P^{(j - 1)}(z) = (z - \alpha)S(z)$, has multiplicity $n + 1$.

    We have that $P^{(k - 1)}(z)$ has root $\alpha$ with multiplicity 1, so $P^{(0)}(z)$ has root $\alpha$ with multiplicity $1 + k - 1 = k$, as desired.
  \end{proof}
\end{enumerate}

\end{document}
