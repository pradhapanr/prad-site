\documentclass[11pt, letterpaper]{article}
\usepackage{fullpage}
\usepackage{amsmath,amsthm,amsfonts,amssymb,amscd}
\usepackage{lastpage}
\usepackage{enumerate}
\usepackage{fancyhdr}
\usepackage{mathrsfs}
\usepackage{enumitem} % \setlist
% for \imageans: float for [H] so the figure floats
\usepackage{graphicx}
\usepackage{adjustbox}
\usepackage{float} 

\setlength{\parindent}{0.25in}
\setlength{\parskip}{0.05in}
% indent paragraphs in list
\setlist{  
  listparindent=\parindent,
  parsep=0pt,
}

% Include graphics in answer
\newcommand{\imageans}[1]
{%
    \begin{figure}[H]
        \centering
        \includegraphics[width=0.4\linewidth]{#1}
    \end{figure}
}

% comments inside align environnment
\newcommand{\comment}[1]{%
  \text{\phantom{(#1)}} \tag{#1}
}
\newtheorem{theorem}{Theorem}
\newtheorem{lemma}{Lemma}
% Cases for Proof environment
\newlist{pcases}{enumerate}{1}
\setlist[pcases]{
  label=\underline{Case~\arabic*}:\protect\thiscase.~,
  ref=\arabic*,
  align=left,
  labelsep=0pt,
  leftmargin=0pt,
  labelwidth=0pt,
  parsep=0pt
}
\newcommand{\case}[1][]{%
  \if\relax\detokenize{#1}\relax
    \def\thiscase{}%
  \else
    \def\thiscase{~#1}%
  \fi
  \item
}

% Edit these as appropriate
\newcommand\course{Math 3122}
\newcommand\hwtitle{HW 3}                  
\newcommand\name{David Tran}
\newcommand\studentid{251169871}

\fancypagestyle{firststyle}
{
    \headheight 35pt
    \lhead{\name}
    \lhead{\name\\\studentid}
    \chead{\textbf{\LARGE \hwtitle}}
    \rhead{\course \\ \today}
    \lfoot{}
    \cfoot{}
    \rfoot{\small\thepage}
    \headsep 1.5em
}

\DeclareUnicodeCharacter{2212}{-}
\begin{document}

\thispagestyle{firststyle}

\setlist[enumerate]{leftmargin=*} % remove enuemrate indentation

\begin{enumerate}
  \item \begin{proof}
    Define $\epsilon(\eta) = \frac{f(z) - f(\eta)}{z - \eta} - f'(z)$ such that $\lim_{\eta \to z} \epsilon(z) = 0$ and $\delta(\zeta) = \frac{g(w) - g(\zeta)}{w - \zeta} - g'(w)$ such that $\lim_{\zeta \to w} \delta(\zeta) = 0$. Then, from the definition of $\delta$, setting $w = f(z)$ and $\zeta = f(\eta)$, we have
      $g(f(z)) - g(f(\eta)) = (\delta(f(\eta)) + g'(f(z)))(f(z) - f(\eta))$. Dividing both sides by $z - \eta$, we have 
      \begin{align*}
        \frac{g(f(z)) - g(f(\eta))}{z - \eta} = (\delta(f(\eta)) + g'(f(z)))\frac{(f(z) - f(\eta))}{z - \eta}
      \end{align*}
      when $z \neq \eta$. So,
      \begin{align*}
        \lim_{\eta \to z}\frac{g(f(z)) - g(f(\eta))}{z - \eta} 
        &= \lim_{\eta \to z}(\delta(f(\eta)) + g'(f(z)))\frac{(f(z) - f(\eta))}{z - \eta} \\
        &= g'(f(z)) f'(z)
      \end{align*}
      where the first equality holds since $f$ is continuous and $\eta \to z$ implies $f(\eta) \to f(z)$, thus $\lim_{\eta \to z}\delta(f(\eta)) = 0$, by definition of $\delta$.
    \end{proof}
  \item By the Cauchy-Riemann equations, $u_x = 2x = v_y$ and $u_y = -2y = -v_x$. So $v(x, y) = 2xy$. So, all such analytic functions have the form $f = x^2 - y^2 + i(2xy) + C$, where $C$ is any constant.
  
  \item \begin{enumerate}
    \item \begin{proof}
      We have $u = e^x \cos y$ and $v = e^x \sin y$, so
      $u_x = e^x \cos y = v_y$ and $u_y = -e^x \sin y = -v^y$. So $e^x$ is entire.
    \end{proof}

    \item \begin{proof}
      \begin{align*}
        e^{z_1 + z_2} &= e^{x_1 + x_2} (\cos (y_1 + y_2) + i\sin(y_1 + y_2)) \\
        &= e^{x_1 + x_2}(\cos y_1 \cos y_2 - \sin y_1 \sin y_2 + i\sin y_1 \cos y_2 + i\cos y_1 \sin y_2) \\
        &= e^{x_1} (\cos y_1 + i \sin y_1) e^{x_2} (\cos y_2 + i \sin y_2) \\
        &= e^{z_1} e^{z_2}
      \end{align*}
    \end{proof}
  \end{enumerate}

  \item \begin{enumerate}
    \item 
      $2 \sin z \cos z
      = 2 \left( \frac{1}{2i} (e^{iz} - e^{-iz}) \frac{1}{2} (e^{iz} + e^{-iz}) \right)
      = \frac{1}{2i}(e^{2iz} - e^{-2iz})
      = \sin 2z$.

    \item
      $\sin^2 z + \cos^2 z
      = -\frac{1}{4} (e^{iz} - e^{-iz})^2 + \frac{1}{4} (e^{iz} + e^{-iz})^2
      = \frac{1}{4}(e^{2iz} + 2 + e^{-2iz} - (e^{2iz} - 2 + e^{-2iz}))
      = 1$.

    \item $
      (\sin z)' = \left(\frac{1}{2i}(e^{iz} - e^{-iz}) \right)'
      = \frac{1}{2i} (ie^{iz} + ie^{-iz})
      = \frac{1}{2} (e^{iz} + e^{-iz})
      = \cos z$.
  \end{enumerate}

  
\end{enumerate}

\end{document}
