\documentclass[11pt, letterpaper]{article}
\usepackage{fullpage}
\usepackage{amsmath,amsthm,amsfonts,amssymb,amscd}
\usepackage{lastpage}
\usepackage{enumerate}
\usepackage{fancyhdr}
\usepackage{mathrsfs}
\usepackage{enumitem} % \setlist
% for \imageans: float for [H] so the figure floats
\usepackage{graphicx}
\usepackage{adjustbox}
\usepackage{float} 

\setlength{\parindent}{0.25in}
\setlength{\parskip}{0.05in}
% indent paragraphs in list
\setlist{  
  listparindent=\parindent,
  parsep=0pt,
}

% Include graphics in answer
\newcommand{\imageans}[1]
{%
    \begin{figure}[H]
        \centering
        \includegraphics[width=0.4\linewidth]{#1}
    \end{figure}
}

% comments inside align environnment
\newcommand{\comment}[1]{%
  \text{\phantom{(#1)}} \tag{#1}
}
\newtheorem{theorem}{Theorem}
\newtheorem{lemma}{Lemma}
% Cases for Proof environment
\newlist{pcases}{enumerate}{1}
\setlist[pcases]{
  label=\underline{Case~\arabic*}:\protect\thiscase.~,
  ref=\arabic*,
  align=left,
  labelsep=0pt,
  leftmargin=0pt,
  labelwidth=0pt,
  parsep=0pt
}
\newcommand{\case}[1][]{%
  \if\relax\detokenize{#1}\relax
    \def\thiscase{}%
  \else
    \def\thiscase{~#1}%
  \fi
  \item
}

% Edit these as appropriate
\newcommand\course{Math 3122}
\newcommand\hwtitle{HW 1}                  
\newcommand\name{David Tran}
\newcommand\studentid{251169871}

\fancypagestyle{firststyle}
{
    \headheight 35pt
    \lhead{\name}
    \lhead{\name\\\studentid}
    \chead{\textbf{\LARGE \hwtitle}}
    \rhead{\course \\ \today}
    \lfoot{}
    \cfoot{}
    \rfoot{\small\thepage}
    \headsep 1.5em
}

\DeclareUnicodeCharacter{2212}{-}
\begin{document}

\thispagestyle{firststyle}

\setlist[enumerate]{leftmargin=*} % remove enuemrate indentation

\begin{enumerate}
  \item \begin{enumerate}
    \item $\frac{3}{20} - \frac{1}{20}i$ (multiplying the numerator and the denominator by the conjugate)
    \item $\frac{4 + 7i}{1 - i} = \frac{(4 + 7i)(1 + i)}{2} = -\frac{3}{2} + \frac{11}{2}i$
    \item Note that $\theta = \operatorname{Arg} z = \frac{\pi}{6}$ and $r = \vert z \vert = 1$, so $z^4 = r^n\operatorname{cis}(4 \theta) = \cos(\frac{2\pi}{3}) + i\sin(\frac{2\pi}{3}) = -\frac{1}{2} + i \frac{\sqrt 3}{2}$ 
    \item $-1 + 0i, 0 - 1i, 1 + 0i, 0 + i, -1 + 0i, \dots$,  
  \end{enumerate}

  \item Let $z_1 = a_1 + b_1i$, $z_2 = a_2 + b_2i$, $z = a + bi$.
  \begin{enumerate}
    \item $\overline{z_1 + z_2} = \overline{(a_1 + a_2) + (b_1 + b_2)i} = (a_1 + a_2) - (b_1 + b_2)i = a_1 - b_1i + a_2 - b_2i = \overline z_1 + \overline z_2$.
    \item $\overline{z_1z_2} = \overline{a_1a_2 - b_1b_2 + (a_1b_2 + a_2b_1)i} = a_1a_2 - b_1b_2 - (a_1b_2 + a_2b_1)i = (a_1 - b_1i)(a_2 - b_2i) = \overline z_1 \overline z_2$
    \item $\overline{P(z)} = \overline{\sum_{k = 0}^n a_k z^k} = \sum_{k = 0}^n \overline{a_k z^k} = \sum_{k = 0}^n a_k \overline{z^k} = \sum_{k = 0}^n a_k \overline z^k = P(\overline z)$ (by (a) and (b))
    \item $\overline{\overline z} = \overline{a - bi} = a + bi = z$
  \end{enumerate}

  \item \begin{proof}
    Suppose $P(z) = 0$. Then by (2c), $P(\overline z) = \overline{P(z)} = \overline 0 = 0.$ Now suppose $P(\overline z) = 0$. Then by (2c), $\overline{P(z)} = 0$, so $\overline{\overline{P(z)}} = P(z) = 0$ by (2d).
  \end{proof}

  \item \begin{proof}
    Let $\epsilon > 0$. Suppose the series converges to $L$, then there is some $N \in \mathbb N$ such that $\vert s_{n - 1} - L \vert < \epsilon/2$ and $\vert s_n - L \vert < \epsilon/2$. Then, $\vert z_n \vert = \vert s_n - s_{n - 1} \vert = \vert (s_n - L) + (L - s_{n - 1}) \vert \leq \vert s_n - L \vert + \vert s_{n - 1} - L \vert < \epsilon$. So $\lim_{n \to \infty} z_n = 0$.
  \end{proof}

  \item \begin{proof}
    ($\rightarrow$) Suppose $S$ is closed. Let $z_0 \in \mathbb C$ and $(z_n) \subseteq S$ such that $\lim_{n \to \infty} z_n = z_0$. Suppose for the sake of contradiction that $z_0 \not\in S$. Then $z_0 \in \mathbb C \setminus S$, which is open. Choose $\epsilon > 0$ such that $D(z_0; \epsilon) \subseteq C \setminus S$. So, $\forall n \in \mathbb N$, $\vert z_n - z_0 \vert \geq \epsilon$, contradicting the assumption that $\lim_{n \to \infty} z_n = z_0$.

    $(\leftarrow)$ Assume the right side is true and suppose for the sake of contradiction that $S$ is not closed. Then $\mathbb C \setminus S$ is not open. So, choose an $z_0 \in \mathbb C \setminus S$ such that $D(z_0; 1/n) \not\subseteq \mathbb C \setminus S$ for all $n \in \mathbb N$. Then, the sequence $(z_n)$ such that $z_n \in D(z_0; 1/n) \cap S$ is in $S$ and converges to $z_0$ but $z_0 \not\in S$; a contradiction.
  \end{proof}

  \item \begin{proof}
    Suppose $S$ is polygonally connected. Assume for the sake of contradiction that $S$ is disconnected. Then, there exists open sets $A, B$ such that $S \subseteq A \cup B$, $A \cap B = A \cap S = B \cap S = \emptyset$. Let $a \in A \cap S$ and $b \in B \cap S$. Since $S$ is polygonally connected, then $a$ and $b$ can be connected with a finite union of line segments $[z_0, z_1], \dots, [z_{n - 1}, z_n] \subseteq S$ with $a = z_0$ and $b = z_n$. Then, there must be some point $z$ along the path such that for every disk $D(z; \epsilon)$, $\epsilon > 0$, $D \not\subseteq A$ and $D \not \subseteq B$, since otherwise the line path could not cross between the disjoint sets $A$ to $B$ (i.e., $z_0 \in A$ and $z_n \in B$). But then, by the openness of $A, B$, $z \not\in A$ and $z \not\in B$, contradicting that $S \subseteq A \cup B$.
  \end{proof} 
  
  \item \begin{proof}
    We have $\vert P(z) \vert = \left\vert \sum_{k = 1}^n a_k z^k \right\vert \geq \sum_{k = 1}^n \vert a_k \vert \vert z^k \vert = \vert z^n \vert (\vert a_n \vert \vert z^0 \vert + \vert a_{n - 1} \vert \vert z^{-1} \vert + \dots + \vert a_0 \vert \vert z^{-n} \vert) \to \vert a_n \vert \vert z^n \vert$ as $z \to \infty$, since each of terms with $z$ to a negative power vanish, and $\vert a_n \vert \vert z^0 \vert = \vert a_n \vert$ for all $z \in \mathbb C$. Since $n \geq 1$, then $\vert z^n \vert \to \infty$ as $z \to \infty$, so $P(z) \to \infty$.
  \end{proof}
\end{enumerate}

\end{document}
