\documentclass[11pt, letterpaper]{article}
\usepackage{fullpage}
\usepackage{amsmath,amsthm,amsfonts,amssymb,amscd}
\usepackage{lastpage}
\usepackage{enumerate}
\usepackage{fancyhdr}
\usepackage{mathrsfs}
\usepackage{enumitem} % \setlist
% for \imageans: float for [H] so the figure floats
\usepackage{graphicx}
\usepackage{adjustbox}
\usepackage{float} 

\setlength{\parindent}{0.25in}
\setlength{\parskip}{0.05in}
% indent paragraphs in list
\setlist{  
  listparindent=\parindent,
  parsep=0pt,
}

% Include graphics in answer
\newcommand{\imageans}[1]
{%
    \begin{figure}[H]
        \centering
        \includegraphics[width=0.4\linewidth]{#1}
    \end{figure}
}

% comments inside align environnment
\newcommand{\comment}[1]{%
  \text{\phantom{(#1)}} \tag{#1}
}
\newtheorem{theorem}{Theorem}
\newtheorem{lemma}{Lemma}
% Cases for Proof environment
\newlist{pcases}{enumerate}{1}
\setlist[pcases]{
  label=\underline{Case~\arabic*}:\protect\thiscase.~,
  ref=\arabic*,
  align=left,
  labelsep=0pt,
  leftmargin=0pt,
  labelwidth=0pt,
  parsep=0pt
}
\newcommand{\case}[1][]{%
  \if\relax\detokenize{#1}\relax
    \def\thiscase{}%
  \else
    \def\thiscase{~#1}%
  \fi
  \item
}

% Edit these as appropriate
\newcommand\course{Math 3124}
\newcommand\hwtitle{HW 9}                  
\newcommand\name{David Tran}
\newcommand\studentid{251169871}

\fancypagestyle{firststyle}
{
    \headheight 35pt
    \lhead{\name}
    \lhead{\name\\\studentid}
    \chead{\textbf{\LARGE \hwtitle}}
    \rhead{\course \\ \today}
    \lfoot{}
    \cfoot{}
    \rfoot{\small\thepage}
    \headsep 1.5em
}

\DeclareUnicodeCharacter{2212}{-}
\begin{document}

\thispagestyle{firststyle}

\setlist[enumerate]{leftmargin=*} % remove enuemrate indentation

\begin{enumerate}
  \item \begin{proof}
    Suppose $f$ is an entire one-to-one function. If $f$ is a polynomial, then by the Fundamental Theorem of Algebra, it has $n$ zeros (counting multiplicity), where $n$ is the degree of $f$. Since $f$ is one-to-one, it has at most $1$ zero of degree $1$. Thus, $n = 1$, so $f$ is a linear function. It remains to show that $f$ must be a polynomial.

    Suppose $f(z)$ is not a polynomial. Then since $f$ is entire, its Taylor expansion at $z = 0$ converges everywhere and has infinitely many terms. So, the principal part of the Laurent series of $f(1/z)$ has infinitely many negative terms. Note that these expansions must be equivalent by the Uniqueness of Laurent expansions. Thus, $f(1/z)$ has an essential singularity at $z = 0$. By the Casorati-Weierstrass Theorem, $f(1/z)$ maps the deleted neighborhood $D$ of $z = 0$ to a dense subset $f(D) \subseteq \mathbb{C}$. Choose an open set disjoint from $D$, say $U$. By the density of $f(D)$ in $\mathbb C$, there is an element $x \in f(D)$ arbitrarily close to an element in $f(U)$. Since $f(U)$ is open by the open mapping theorem, $x \in f(U)$. So $f(D)$ and $f(U)$ are not disjoint despite $D$ and $U$ being disjoint, so $f$ is not injective; a contradiction. So $f$ is a polynomial.
  \end{proof}

  \item \begin{enumerate}
    \item $z = 0$ and $z = \pm i$ are poles since $1$ and $z^4 + z^2$ are polynomials and thus entire, and $z^4 + z^2 = 0$ at these values. $z = 0$ is a pole of order 2 since it is a zero of order 2 of $z^4 + z^2 = z^2(z^2 + 1)$.
    \item $z = k\pi$ for $k \in \mathbb Z$ are poles since $\cot z = \cos z / \sin z$ and $\sin z = 0$ at these values. And, $\sin z$ and $\cos z$ are entire.
    \item $z = k\pi$ for $k \in \mathbb Z$ are poles since $\csc z = 1/\sin z$ and $\sin z = 0$ at these values. And, $\sin z$ and $1$ are entire.
    \item $z = 1$ is a pole since $z - 1 = 0$ there and $z - 1$ and $\exp(1/z^2)$ are analytic at $z = 1$. $z = 0$ is an essential singularity since it is not a pole since $\exp(1/z^2)$ is analytic at $z = 0$. And, $z = 0$ is not a removable singularity since $\exp(1/z^2)$ is an essential singularity (its Laurent expansion has infitely many terms in its principal part).
  \end{enumerate}

  \item \begin{enumerate}
    \item \begin{align*}
      \frac{1}{z^4 + z^2} = \frac{1}{z^2(z^2 + 1)} = \frac{1}{z^2}\sum_{k = 0}^\infty (-1)^k z^{2k} = \sum_{k = 0}^\infty (-1)^k z^{2k - 2}
    \end{align*}
    \item \begin{align*}
      \frac{\exp(1/z^2)}{z - 1} = - \frac{\exp(1/z^2)}{1 - z} = - \sum_{k = 0}^\infty z^k \cdot \sum_{k = 0}^\infty \frac{1}{k!z^{2k}}
      = - \sum_{k = 0}^\infty \sum_{j = 0}^\infty \frac{z^{j - 2k}}{k!}
    \end{align*}
    \item Note that \begin{align*}
      \frac{1}{z + 2}
      &= \frac{1}{z - 2} \cdot \frac{1}{1 + \frac{4}{z - 2}} \\
      &= \frac{1}{z - 2} \cdot \sum_{k = 0}^\infty \left(-\frac{4}{z - 2}\right)^k \\
      &= \frac{1}{z - 2} + \sum_{k = 1}^\infty \frac{(-4)^k}{(z - 2)^{k + 1}} \\
    \end{align*}
    so
    \begin{align*}
      \frac{1}{z^2 + 4}
      &= \frac{1}{4} \left( \frac{1}{z - 2} - \frac{1}{z + 2}\right) \\
      &= -\frac{1}{4} \sum_{k = 1}^\infty \frac{(-4)^k}{(z - 2)^{k + 1}} \\
      &= \frac{1}{4} \sum_{k = 2}^\infty \frac{(-4)^k}{(z - 2)^{k}} \\
    \end{align*}
  \end{enumerate}
\end{enumerate}

\end{document}
